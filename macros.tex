\newtheorem{p}{Problem}
\newenvironment{s}{%\small%
\begin{trivlist} \item \textbf{Solution}. }{%
\hspace*{\fill} $\blacksquare$\end{trivlist}}%


\newcommand{\topic}[1]{
\newpage
{\noindent\Huge\bf 
  \\[0.5\baselineskip]
  {\fontfamily{cmr}\selectfont  #1 \addcontentsline{toc}{section}{#1}}
}
}
\newcommand{\estopic}[1]{
\\[2\baselineskip] % Title
{ 
  {\bf \fontfamily{cmr}\selectfont #1 \addcontentsline{toc}{subsection}{#1}}\\
  {\textit{\fontfamily{cmr}\selectfont    \today}}
} \hfill {\large \textsc{Michael Zhao}} % Author name
\\[1.4\baselineskip]
}


\makeatletter
\def\thm@space@setup{%
  \thm@preskip=0.5em\thm@postskip=\thm@preskip%
}
\makeatother
% Unnamed theorem without counter
\newtheorem*{theorem}{Theorem}
\newtheoremstyle{named}{}{}{\\itshape}{}{\bfseries}{.}{.5em}{\thmnote{#3's }#1}
% Named theorem without counter
\theoremstyle{named}
\newtheorem*{namedtheorem}{Theorem}
% Other theorem environments
\theoremstyle{plain}
\newtheorem{thm}{Theorem}[section]
\newtheorem{prop}[thm]{Proposition}
\newtheorem{lem}[thm]{Lemma}
\newtheorem{cor}[thm]{Corollary}
\theoremstyle{definition}
\newtheorem{defn}[thm]{Definition}
\newtheorem{ex}[thm]{Exercise}
\newtheorem{exmpl}[thm]{Example}
\newtheorem{fact}{Fact}
\theoremstyle{remark}
\newtheorem{rmk}[thm]{Remark}

% Grothendieck group
\newcommand{\ggroup}{K_0}

% GL_n
\newcommand{\gl}[2]{\mathrm{GL}_{#1}({#2})}
% SL_n
\renewcommand{\sl}[2]{\mathrm{SL}_{#1}({#2})}
% SO_n
\newcommand{\so}[2]{\mathrm{SO}_{#1}({#2})}
% Sp_n
\renewcommand{\sp}[2]{\mathrm{Sp}_{#1}({#2})}
% G_m
\newcommand{\gm}{\mathbb{G}_m}

% Trace
\newcommand{\tr}{\operatorname{tr}}

% Conjugation
\newcommand{\conj}[1]{\overline{#1}}
% Class group
\newcommand{\cl}[1]{{\mathrm{Cl}{(#1)}}}
% Order of a field
\newcommand{\roi}[1]{\mathcal{O}_{#1}}

% Galois group of L/K
\newcommand{\gal}[2]{\mathrm{Gal}{(#1 / #2)}}

% Rationals
\newcommand{\rats}{\mathbb{Q}}
% Real numbers
\newcommand{\reals}{\mathbb{R}}
% Nonnegative real numbers
\newcommand{\nnreals}{\mathbb{R}^{\geq{0}}}
% Complex numbers
\newcommand{\cmplx}{\mathbb{C}}
% Integers
\newcommand{\ints}{\mathbb{Z}}
% p-adic rationals
\newcommand{\Qp}{\rats_p}
\newcommand{\Zp}{\ints_p}
% matrices
\newcommand{\bp}{\begin{pmatrix}}
\newcommand{\ep}{\end{pmatrix}}
%fonts
\newcommand{\mc}{\mathcal}
\newcommand{\mf}{\mathfrak}


% Algebraic number theory
%  Prime ideals
\newcommand{\prim}{\mathfrak{p}} % \prime is the symbol: '
\newcommand{\oprime}{\mathfrak{P}}
\newcommand{\oprimealt}{\oprime^o}

% Frobenius map
\newcommand{\frob}[1]{\mathrm{Fr}_{#1}}
% Legendre symbol
\newcommand{\legndr}[2]{\left(\dfrac{#1}{#2}\right)}
% Norm, with the fields understood
\newcommand{\N}[1]{N{(#1)}}

% Integers modulo a given number
\newcommand{\Zmod}[1]{\mathbb{Z}/#1 \mathbb{Z}}
% Units of integers mod n
\newcommand{\Zmodu}[1]{\left(\mathbb{Z}/#1 \mathbb{Z}\right)^\times}

% Langlands dual
\newcommand{\langdual}[1]{{#1}^\vee}

% Category of representations
\newcommand{\rep}{\mathrm{Rep}}

% Character group of a group
\newcommand{\cgroup}[1]{X^\bullet(#1)}

\newcommand{\Hom}{\operatorname{Hom}}
\newcommand{\Ind}{\operatorname{Ind}}
\newcommand{\aut}{\operatorname{Aut}}
\newcommand{\into}{\operatorname{End}}
\newcommand{\Stab}{\text{Stab}}
\newcommand{\lsurj}{\twoheadleftarrow}
\newcommand{\rsurj}{\twoheadrightarrow}
\newcommand\restr[2]{{% we make the whole thing an ordinary symbol
  \left.\kern-\nulldelimiterspace % automatically resize the bar with \right
  #1 % the function
  \vphantom{\big|} % pretend it's a little taller at normal size
  \right|_{#2} % this is the delimiter
}}


\newcommand{\Spec}[1]{\mathrm{Spec}({#1})}


\newcommand{\op}[1]{\operatorname{#1}}
% Affine Grassmannian of a group
\newcommand{\gr}[1]{\mathrm{Gr}_{#1}}

% Category of perverse sheaves
\newcommand{\perv}{\mathrm{Perv}}
% Regular-sized isomorphism.
\newcommand{\iso}{\simeq}

% A large, noticeable quotient
\newcommand{\quotient}[2]{{\raisebox{.2em}{$#1$}
                           \left/\raisebox{-.2em}{$#2$}\right.}}

% Experimental big isomorphism commands.

% Bigger isomorphism (with arrow).
\newcommand{\longiso}{\stackrel{\sim}{\longrightarrow}}

% Very big isomorphism (no arrow).
\makeatletter
\newcommand*{\isomorphism}{%
  \mathrel{%
    \mathpalette\@isomorphism{}%
  }%
}
\newcommand*{\@isomorphism}[2]{%
  % Calculate the amount of moving \sim up as in \simeq
  \sbox0{$#1\simeq$}%
  \sbox2{$#1\sim$}%
  \dimen@=\ht0 %
  \advance\dimen@ by -\ht2 %
  %
  % Compose the two symbols
  \sbox0{%
    \lower1.9\dimen@\hbox{%
      $\m@th#1\relbar\isomorphism@joinrel\relbar$%
    }%
  }%
  \rlap{%
    \hbox to \wd0{%
      \hfill\raise\dimen@\hbox{$\m@th#1\sim$}\hfill
    }%
  }%
  \copy0 %
}
\newcommand*{\isomorphism@joinrel}{%
  \mathrel{%
    \mkern-3.4mu %
    \mkern-1mu %
    \nonscript\mkern1mu %
  }%
}
\makeatother

\newcommand{\arrowcircle}[1][]{%
  \begin{tikzpicture}[#1]
    \draw[right hook->] (0,0ex) -- (2em,0ex);
    \draw (1em,0ex) circle (0.5ex);
  \end{tikzpicture}%
}

\newcommand{\catname}[1]{{\textsf{{#1}}}}
\newcommand{\Ab}{\catname{Ab}} % category of abelian groups
\newcommand{\A}{\mathcal{A}}
\newcommand{\B}{\mathcal{B}}
\newcommand{\RHom}{\mathrm{RHom}}
\newcommand{\qiso}{\stackrel{\longrightarrow}{\sim}}
\newcommand{\Sh}{\catname{Sh}}
\newcommand{\iHom}{\mathcal{H}om}
\newcommand{\id}{\mathrm{id}} % identity map
\newcommand{\im}{\mathrm{im} \ } % image
\newcommand{\comp}{\circ} % composition
\renewcommand{\to}{\rightarrow} %maps to
\newcommand{\Z}{\mathbb{Z}} % integers
\newcommand{\D}{\Delta} % diagonal morphism
\newcommand{\tp}{\otimes} % tensor product
\newcommand{\fp}{\times} % fibred product
\newcommand{\imp}{\Rightarrow} % implication sign
\newcommand{\rimp}{\Leftarrow} % reverse implication sign
\renewcommand{\sp}{\mathrm{sp} \ }
\newcommand{\closure}{\overline}
\newcommand{\spesh}{\rightsquigarrow}
\newcommand{\bi}[1]{\emph{\textbf{#1}}}

\renewcommand{\O}{\mathcal{O}}
\newcommand{\F}{\mathcal{F}}
\newcommand{\G}{\mathcal{G}}
\newcommand{\I}{\mathcal{I}}

\renewcommand{\ep}{\varepsilon}

\newcommand{\fm}{\mathfrak{m}}

\newcommand{\emptylineskip}{\vspace{\baselineskip}}
